\textit{Dangerous to the hand and pleasing to behold, the thorns arise in
whorls. Pockmarked, but sometimes smooth, the sites of the thorns' emergence
constitute a ledger of recurrent violation. Read in the blood of pinpricks the
thorns transfigure our fairytales; reread in metal, the thorns multiply
according to a regime of walls and barbed wire. The American West, children
read at their desks, developed in a proliferation of these fences across which
cattle can not pass, inscribing in the movements of animals the decisions of an
economics that would come to totalize both them and us alike. If only these
ribbons~---~and with them their thorns~---~had expired in the sun of our
rangeland. If only the bodies confined at these folds were of another place,
another type. If only the thorns weren't now.}

\textbf{Interpretation.} The music is structured according to an interpolation
of voices. The interpretation of this type of intercalative polyphony should
prioritize the connectedness of each voice: one voice interrupting another
should convey a type of curtailment; each voice's reappearance should effect
the sudden return of a music hidden from view. \textbf{Color.} Some indications
of articulation are provided in the first half of the score. But the choice of
both instrumental color and dynamics prior to measure 182 is left very largely
to the interpreter. Select combinations of playing technique and dynamic to
convey the polyphony of the first half of the piece as fantastically as
possible. Note, however, that explicit indications of color and dynamic appear
starting at measure 182; these should be followed exactly until the end of the
score. \textbf{Tamburo trills.} Beginning at measure 78 the score calls for
tamburo trills. Play these by tapping the pads of the index and middle fingers
in rapid alternation. From measure 182 the score varies the number of fingers
used to execute these trills; add the ring finger for 3-finger trills and add
the little finger for 4-finger trills. Play all tamburo trills with the fleshy
pads of the fingers and never with the fingernails or with the knuckles; all
such trills are relatively quiet. \textbf{Rasgueado.} Beginning at measure 214
the score introduces two types of rasgueado: ``knuckle rasgueado'' and ``nail
rasgueado.'' The second of these~---~nail rasgueado~---~is the familiar
four-finger strumming technique originating in flamenco. Knuckle rasgueado is a
modification of this technique played with the knuckles instead of fingernails.
Knuckle rasgueado functions as a type of muted or veiled version of the usual
nail rasgueado. Note that the score requests gradual transitions of color
between these two types of rasgueado and the tamburo trills described above.
\textbf{Scordatura.} At the start of the piece the guitar is tuned as usual.
Raise string \ding{173} a quartertone at measure 124 and leave the scordatura
in place for the remainder of the piece; this means that 12-ET pitches played
on string \ding{173} after measure 124 will sound a quartertone higher than
written, with the selection of pitches to be played on string \ding{173} left
to the performer. \textbf{Amplification.} The piece may be amplified according
to the judgement of the performer.
