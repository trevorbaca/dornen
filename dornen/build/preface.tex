\documentclass[10pt]{article}
\usepackage[utf8]{inputenc}
\usepackage[papersize={17in, 11in}]{geometry}
\usepackage[absolute]{textpos}
\TPGrid[0.5in, 0.25in]{23}{24}
\usepackage{palatino}
\parindent=0pt
\parskip=12pt
\usepackage{nopageno}
\begin{document}

\begin{textblock}{23}(0, 1)
\center \huge PREFACE
\end{textblock}

\begin{textblock}{23}(0, 3)

Dangerous to the hand and pleasing to behold, the thorns arise in whorls.
Pockmarked, sometimes smooth, the thorns' sites of emergence record in their
growth a history of recurrent violation. Read in blood -- a pinprick, nothing
more -- the thorns migrate from the garden and into our fairytales; reread in
metal the thorns' shapes metastisize according to a regime of walls and barbed
wire. The American West, children read at their desks, developed in the
profileration of these fences across which cattle could not move, inscribing in
the movements of animals the decisions of an economics that would come, with
time, to totalize animals and their people alike. If only these ribbons of
wire, and with them their thorns, had expired in the sun and the clay-caked
grasses of our rangeland. If only the bodies confined in the folds of these
wires were of another place, a not-now, a never-then.

\end{textblock}

\end{document}