\documentclass[11pt]{article}
\usepackage{amssymb}
\usepackage[left=0.85in, top=1in, right=0.85in, bottom=1in, nohead]{geometry}
\usepackage{graphicx}
\usepackage[utf8]{inputenc}
\usepackage{mathtools}
\usepackage{parskip}
\usepackage{url}
\sloppy
%\title{Statement \\ of \\ Teaching Philosophy}
\author{}
\date{}
\begin{document}
%\maketitle
\pagenumbering{gobble} 
\thispagestyle{empty}
\setlength{\parindent}{0cm}

30 September 2017

Dear Nico,

When you asked me to write for you, the thorns were with me right away.

I'd had German \textit{Dorn} (etymologically English \textit{thorn}; Flemish \textit{Doorn}; the in its pre-Christian form, must have meant something like `stiff' or `stern') with me since at least the time of \textit{L'archipel du corps}. I had to imagine \textit{L'archipel} and its four instruments (flute, guitar, accordion, percussion) broken into smaller groups --- solos, duos, trios --- each with the potential for development into some future, independent work. The `archipelago' of \textit{L'archipel} was, phenomenologically, the impossibility of experiencing the body as a whole, as a body; the bracketing of external experience had convinced me that perceptions of the body as such are remarkable because of the degree to which they are only ever broken-off fragments of \dots something \dots though what that something is (naively, the body as wholeness) presents itself within the phenomenological frame not at all. How this can be -- how we come to understand the body as the site of reflection that, directed back on itself, finds itself always only in pieces -- is difficult to think, and perhaps not possible at all (if what blocks the thought turns out to the be the lack of a position from which both the body and its constituent phenomena can be serially apprehended).

And it was precisely the difficulty of this thought --- the crack, as it were, between the never-present whole and the manifest parts --- that motivated the hundreds of disjunct rhythms that present so much of the musical difficulty of the quartet: the phenomenological `flicker' of body in its self-directed reflections found its way into the local details of how material moved in that piece --- into the piece's approach to rhythm, that is --- quite apart from the forward- and backward)directed motion of time in that music. (\textit{L'archipel} was, of course, also the introduction of the machinist's screw into my music, which was to come back in such an important position in \textit{Dornen}; more on that below.)

\textit{L'archipel} showed me that the imagined collection of the music's smaller groupings --- the unwritten solos, duos, trios --- were conceptually valid: the conceit of the body (taken through its phenomenological considerations up to the breaking-point of those considerations and, then, the subsequent discovery of the music's movements in time) functioned as a musical conceit precisely because of the concepts's aggregate parts and the ways in which we only ever rediscover wholeness, undiminished, in them. (There are sketches, still, in a box in the closet for \textit{Las manos m\'{a}gicas}, which would have come immediately after, for alto flute and guitar; but life intervened.)

The other pieces in the work-cycle remain to be written --- it's always a matter of finding the right moments for these things to come to life --- and it was in this anticipatory readiness that your invitation found me when you asked for the music that would become \textit{Spiel der Dornen}; the thorns, as I said, were with me right away.

The bridge between the manifest body (\textit{L'archipel du corps}), the hands (\textit{Las manos m\'{a}gicas}) and \textit{Dornen}'s thorns came from a reading of German \textit{Dorn} as English \textit{spine} (in the second sense), which allowed me to re-think the motivations for the machinist's screw in \textit{L'archipel} (and, later, \textit{Dornen}): the appearance of the screw --- of threaded metal --- in the quartet had come from my imaginings of the wrapped and braided edges of cactus' spines; its not the sharp point of cactus' spines that makes them dangerous so much as the rifling of their edges (still plenty of cactus in Texas, even in the suburban parts I grew up in) which is what causes the spines to lodge in the birds and animals they pierce. The rifled treatment of the metal screw gives it the sound it has --- variegated, other-worldly --- when drawn in various ways over the guitar's strings. And though cactus are not unfamiliar in contemporary music (John Cage, Mark Andr\'{e}) it wasn't until you and I met in Maastricht to play \textit{L'archipel} that I heard the cactus-spine-cum-twisted-screw as pregnant for radically different contextualization in a different piece; the piece --- \textit{Dornen} --- that would, in the end, be a reflection on the literalness of the spine --- your spine, in fact --- together with the prayer for wholeness that that music (and my wishes for you at the time I was writing it) enwraps.

\textit{Dornen} is built in halves. The first of these --- rhythmically intricate, elaborately worked-out in the manipulation of the endless strings of pitches that garland the music about itself --- is interrupted in exactly three places by what we hear at first as a type of peripheral --- or even subterranean --- glow. The successive first appearances of that tamburo trill --- product of the fleshy soft pads of your first two fingers only, quieter than the music that comes before and after, seemingly projected from elsewhere --- portend everything that is to happen in the second part of the piece: the waves and waves of time that are articulated by the unbroken effort of your rasgueado (sinking, as it does, down by step only three times in almost seven minutes) that you are always able --- somehow --- to articulate as a single seven-minute thought, unimaginably difficult to do.

This rasgueado --- at once so important a part of the guitar's history, an arrow passing through Iberia directly to North Africa --- is no token of any (music) historicism. We built that rasgueado together --- the dozens of changes between two, three, four fingers together with the dozens of exchanges of pads, nails, knuckles --- in a search for the ``ever-increasing envelopes of time" (as it came out in the piece's program note) that effect --- fundamentally --- the being-towards-the-world that \textit{Spiel der Dornen} suggests:

Whatever the limits of thought to apprehend, in their fullness, lived experience of the one (Plotinus) and the amalgam of perceived (Augustine) and perceiver (Lebniz) that is the only light by which we come read our own appearance amid the onslaught of time, the thorns were multiple --- they are always many, never just one thorn --- and they ring whatever time we understand ourselves to have (or even not to have). The thorns were here before. And they will be here after.

When it came time write the inscription at the head of the score, it took me several nights:

\textit{Dangerous to the hand and pleasing to behold, the thorns arise in whorls. Pockmarked, but sometimes smooth, the sites of the thorns? emergence constitute a ledger of recurrent violation. Read in the blood of pinpricks the thorns transfigure our fairytales; reread in metal, the thorns multiply according to a regime of walls and barbed wire. The American West, children read at their desks, developed in a proliferation of these fences across which cattle can not pass, inscribing in the movements of animals the decisions of an economics that would come to totalize both them and us alike. If only these ribbons --- and with them their thorns --- had expired in the sun of our rangeland. If only the bodies confined at these folds were of another place, another type. If only the thorns weren?t now.}

It was the last bit of work to be done --- you had had the music for several weeks --- and by the time it came to write the words, the thorns had multiplied (in the way thorns do) and the barbed wire I introduced to carry some of the work of the music's meaning called to mind both the rangeland of Texas and the American southwest and also, of course, the German camps (and the years of \textit{Mauer und Stacheldraht} that would follow). This opening-up of the image in its fullness is what caused me to add to the commentary of score's inscription, later, for the program notes:

\textit{The piece is about figuration: the bits and pieces of the music that present themselves to consciousness as the component elements of (nonfunctional) rhetoric, neither small enough to constitute the music?s ultimate components nor big enough to complete its perceptual moments. History hides here. Because it is figuration --- not harmony --- inside of which the past inheres and from the footholds of which it reaches to drag us back. The tuning of the instrument is left as it was, until the tuning of the instrument is disfigured. The colors of the instrument are taken from its past, until the colors of the instrument are distended. The world I understand asserts the strictures of a dichotomy I didn?t ask for. So no resolution here. At best, ever-increasing envelops of time.}

\textit{To return to the of image of the inscription: there was never a thorn, a spine, a prickle anywhere in the spinning wheel or its parts. The distaff is rounded and the sleeping beauty legend carries in it the fears of our technology wrapped in a lie about its parts. But barbed wire is real enough. And whatever may have been the intention of cattlemen, the present exigencies have to do instead with the function of the state. At the close of the second war, it wasn?t animals but people the American government released from camps it built for its purposes in California.}

When we premiered the piece together in Darmstadt, Viktor Orb\'{a}n in Hungary (and other right-wing leaders elsewhere in Europe and the US) was making his case against newcomers, against asylum-seekers, against refugees quite literally in barbed wire. And so the unremitting reality of the cycle continued, played out in the realities of people in need. I couldn't, of course, point to Europe first in the explicit notes to accompany the music; I was born in California, and there were, and remain, plenty of thorns to adduce in the shapes of our history, no matter in what corners of it we find ourselves.

And so there is one telling of \textit{Dornen}.

Whether we locate \textit{Dornen} relative to the end of Passiontide or to an eternal return (Nietzsche) we must accept as the fullness a time that we can not understand, depends on what we understand of the thorns' multiplicity: the thorns were here before; the thorns will be here after; what it is that the thorns ring is what we understand (or do not understand) as the present, replete with the flickering sensations of a body arises from parts we can not know.

\vspace {10 mm}

\hfill--- Trevor

\end{document}